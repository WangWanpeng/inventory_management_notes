\section{Appendix}
\label{sec:appendix}

\Opensolutionfile{ans}

For the rest of the course we need some important notation. \nvf{to do}
\begin{exercise}
What is the meaning of $\sum_{i=1}^n x_i$ and why is $h \sum_{i=1}^n x_i = \sum_{i=1}^n (h x_i)$, where $h$ is some number?
  \begin{solution}
    \begin{align*}
      \sum_{i=1}^3 x_i &= x_1 + x_2 + x_3 \\
      \sum_{i=1}^n x_i &= x_1 + x_2 + \cdots +x_n \\
     h \sum_{i=1}^n x_i &= h(x_1 + x_2 + \cdots +x_n) = h x_1 + \cdots + h x_n = \sum_{i=1}^n h x_i.
    \end{align*}
  \end{solution}
\end{exercise}

We define the \recall{indicator function} of a claim $A$ as
  \begin{equation*}
    \1{A} =
    \begin{cases}
      1, & \text{if $A$ is true},\\
      0, & \text{if $A$ is not true}.
    \end{cases}
  \end{equation*}
For instance $\1{3<5} = 1$ and $\1{\text{dogs can fly}} = 0$. This function is also easy to implement in excel.

Finally, define
\begin{align}
 x^+ &= \max\{x, 0\} & x^- &= \max\{-x, 0\}.
\end{align}

\begin{exercise}
  Show that 
  \begin{align*}
 x  &= x^+ - x^-, & (x-y)^+ &= x - \min\{x, y\}.
  \end{align*}
  \begin{solution}
    \begin{equation*}
      x = \max\{x, 0\} + \min\{x, 0\} = x^+ - \max\{-x, 0\} = x^+ - x^-.
    \end{equation*}
  \end{solution}
\end{exercise}


Define 
\begin{align}\label{eq:3}
 x^+&=\max\{x,0\}, & x^-&=\max\{-x,0\}.
\end{align}
In the derivations below we heavily use these definitions and the relations of the next two exercises.

\begin{exercise}
Show that
\begin{align}\label{eq:4}
 x&=x^+ - x^- &  (-x)^- &=x^+.
\end{align}
\begin{solution}
Take simple numerical examples to see why it is true. More formally, let us consider two cases: $x\geq 0$ and $x<0$. In the first case, we have $x^+=\max\{x,0\}=x$ and $x^-=\max\{-x,0\}=0$. Thus $x=x-0$ holds. In the second case, we have $x^+=\max\{x,0\}=0$ and $x^-=\max\{-x,0\}=-x$. Thus $x=0-(-x)$ holds. 

For the other equality: $(-x)^- = \max\{- (-x),0\} = \max\{x, 0\} = x^+$. 
\end{solution}
\end{exercise}

\begin{exercise}\label{ex:6}
Show that
\begin{align}
\min\{x,y\} &=  x - (x-y)^+ & (x-y)^+ = x - \min\{x, y\}.
\end{align}
Show this.
\begin{solution}
  \begin{equation*}
    x-(x-y)^+ =
    \begin{cases}
      x-(x-y) = y &\text{ if } x-y\geq 0\\
      x-0 = x &\text{ if } x-y\leq 0.
    \end{cases}
  \end{equation*}
Hence, $x-(x-y)^+ = \min\{x, y\}$. The other result follows from this. 
\end{solution}
\end{exercise}

the \recall{probability mass function} (\recall{pmf}) $f(i) = \P{ D = i}$ is given. Recall that the \recall{distribution function} $F$ and the mass function $f$ are related by
\begin{equation*}
F(i) = f(0)+f(1)+\cdots f(i).
\end{equation*}

\begin{exercise}\label{ex:exp} Let $D$ be a random variable with probability mass function $f$. To help recall the concept of \recall{expection},
write the expected demand $\E{D}$ in terms of a summation.
  \begin{solution}
$\E{D}$ is just a short-hand for
    \begin{equation*}
      \E{D} = \sum_{i=0}^\infty i f(i).
    \end{equation*}
  \end{solution}
\end{exercise}


\begin{exercise}
  Let $f(1)=f(2)\ldots=f(5) = 1/5$. Compute $\theta = \E{D}$, i.e., the expected demand.
  \begin{solution}
    \begin{align*}
      \E{D} &= \sum_{i=1}^5 i f(i) = \frac 1 5+\frac 2 5+\cdots+\frac 55 = 3, \\
    \end{align*}
  \end{solution}
\end{exercise}


\Closesolutionfile{ans}
\opt{solutionfiles}{
\subsection{Solutions}
\input{ans}
}
\clearpage

%%% Local Variables:
%%% mode: latex
%%% TeX-master: "inventory_notes"
%%% End:
