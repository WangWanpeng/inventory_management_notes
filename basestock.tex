

\subsection{Base Stock Model}


\subsubsection{Common notation}
\label{sec:common-notation}

Let $I_t$ denote the on-hand inventory at time $t$, $B_t$ the number
of backorders, $R_t$ the number of out-standing replenishments, and
$X(t-L,t]$ the demand that occurred during the time interval
$(t-L, t]$. Note that here we write $X(t-L, t]$ to represent the
random variable $X$ of the book.


The above random variables satisfy a number of important relations.
First, as for each demand a replenishment order is sent, it must be that
\begin{equation}
  \label{eq:8}
   R_t = X(t-L, t],
\end{equation}
that is, the number of outstanding replenishments at time $t$ equals
all demand that occurred during the previous leadtime.  Hence,
\begin{equation}
  \label{eq:16}
   G(r) = \P\{X\leq r\} = \P(X(t-L, t])).
\end{equation}
Note that the average demand during a lead time is
\begin{equation*}
\theta = \E(X(t-L, t]).
\end{equation*}

Second, as each demand spawns a replenishment, it must be for all time
$t$ that the inventory level $I_t$ plus the number of
replenishments $R_t$ minus all backorders $B_t$ remains
constant. That is, 
\begin{equation*}
I_t + R_t - B_t = \text{ constant}.
\end{equation*}

Third, we do not backorder demand when there is on-hand stock and we
also match backorders (if any) with replenishments as they arrive, it
must hold that
\begin{equation}
  \label{eq:9}
   I_t B_t =0, \text{ for all }  t\geq 0.
\end{equation}
In other words, either the on-hand inventory is $0$ or the
number of backorders is 0.

According to the basestock policy we issue a replenishment order as
soon as the re-order level $r$ is hit. Assuming that at time $t=0$,
there are no outstanding replenishments and no backorders, we can
safely assume that $I_0 = r+1$. The above then implies that
\begin{equation}
  \label{eq:7}
   I_t + R_t - B_t = r+1, \text{ for all }  t\geq 0.
\end{equation}

When dealing with positive leadtimes two concepts are very useful. The \emph{inventory level} $IL_t$ at time $t$ consists of all items on hand or on backlog, i.e.,
\begin{equation}\label{eq:23}
  \IL_t = I_t - B_t.
\end{equation}
The \emph{inventory position} $IP_t$ is the inventory level plus all outstanding replenishments:
\begin{equation*}
  \IP_t = \IL_t + R_t.
\end{equation*}

From (\ref{eq:7}) we conclude that 
\begin{equation}\label{eq:20}
\IP_t =\IL_t + R_t = I_t - B_t +R_t = r+1.
\end{equation}
Thus, for the basestock mode in continuous time, the inventory position is constant. 




\subsubsection{Computing the Service level}

The service level $S(r)$ is defined as the fraction of demand that
perceives, on arrival, a positive stock level. As we assume that demand
occurs in single units, this fraction is therefore equal to the fraction
of demand served from on-hand stock. We also assume that the arrival
process is given by a Poisson process. Therefore, by the PASTA property,
the fraction of demand served from stock is equal to the (long-run)
fraction of time that the inventory level is positive. Hence,
\begin{equation*}
   S(r) = \P\{I_t >0\}.
\end{equation*}

Using that $I_t>0$ at time $t$ implies that $B_t = 0$,
and \ref{eq:7}:
\begin{equation}\label{eq:10}
  \begin{split}
   S(r) &= \P\{I_t >0\} \\
   &= \P\{r+1 + B_t - R_t >0\}, \text{  from  \ref{eq:7}}  \\
   &= \P\{r+1 - R_t >0\}, \text{ as } B_t = 0, \\
   &= \P\{R_t < r+1\} \\
   &= \P\{R_t \leq r\} \\
   & = \P\{X(t-L,t] \leq  r\}, \text{from  \ref{eq:8}}, \\
   &= G(r),  \text{ from \ref{eq:16}} \\
   &=  \sum_{i=0}^{r} g_i.
  \end{split}
\end{equation}
Thus,
\begin{equation}
  \label{eq:13}
   S(r) = G(r) = \sum_{i=0}^r \PP{X(t-L,t] = i},
\end{equation}
and \emph{not} $G(r+1)$ as in Factory Physics.

\subsubsection{Computing the average backorder level}


When does a back-order occur? This happens whenever
\begin{equation}
  \label{eq:11}
   \{B_t > 0\} = \{R_t - r-1>0\} = \{X(t-L, t] - r-1>0\},
\end{equation}
where we use \ref{eq:9}, \ref{eq:7} and \ref{eq:8}. Hence,
\begin{equation*}
   \begin{split}
     B(r) 
   &= \E(B_t) \\
   &= \E(\max\{B_t, 0\}) \\
   &= \E(\max\{R_t - r - 1, 0\}) \\
   &= \E(\max\{X(t-L, t] - r - 1, 0\}) \\
   &= \sum_{i=r+1}^\infty (i- r -1)\P\{X(t-L, t] = i\}\\
   &= \sum_{i=r+1}^\infty (i- r -1)g_i \\
   &= \sum_{i=r+2}^\infty (i- r -1)g_i,
     \end{split}
\end{equation*}
where the last equation follows from the fact that when $i=r+1$,
$i-r-1 =0$. I find the following easier to memorize, hence I use this
in the sequel:
\begin{equation}
  \label{eq:12}
   B(r)  = \sum_{i=r+1}^\infty (i- r -1)g_i.
\end{equation}

For later purposes we prove that
\begin{equation}
  \label{eq:17}
   B(r) = \sum_{i=r+1}^{\infty} \bar G(i)
\end{equation}
where
\begin{equation}
  \label{eq:18}
   \bar G(i) = \P(X>i) = 1 - P(X\leq i) = 1 - G(i).
\end{equation}
Define first the function
\begin{equation*}
   1_{i< j} =
     \begin{cases}
       1, &\text{  if } i < j, \\
   0, &\text{ else},
     \end{cases}
\end{equation*}
so that we can write
\begin{equation*}
  \sum_{j=0}^\infty 1_{j< i-r - 1} = i-r -1.
\end{equation*}
Now, using \ref{eq:12},
\begin{equation}
  \label{eq:19}
  \begin{split}
       B(r) &= 
   \sum_{i=r+1}^\infty (i-r-1) g(i)   \\
   &= \sum_{i=r+1}^\infty\sum_{j=0}^\infty 1_{j < i-r-1}\, g(i)   = 
    \sum_{j=0}^\infty \sum_{i=r+1}^\infty 1_{i > j +r + 1}\, g(i)\\
   &= \sum_{j=0}^\infty \sum_{i=j + r+2}^\infty  g(i) = 
   \sum_{j=0}^\infty \P(X \geq j + r+2)  \\
   &=\sum_{j=0}^\infty \P(X > j + r+1)  \\
   &= \sum_{j=0}^\infty \bar G(j+r+1) =\sum_{j=r+1}^\infty  \bar G(j).
  \end{split}
\end{equation}
Finally, this can be simplied a bit by using that
$\sum_{i=0}^\infty \bar G(i) = \theta$:
\begin{equation}
  \label{eq:119}
  \begin{split}
   B(r) 
   &= \sum_{j=r+1}^\infty  \bar G(j) \\
   &= \sum_{j=0}^\infty  \bar G(j) - \sum_{j=0}^{r} \bar G(j)\\
   &= \theta - \sum_{j=0}^{r} \bar G(j)
  \end{split}
\end{equation}
	   

\subsubsection{Computing the expected inventory level}

Taking expectations at the left and right hand side of \ref{eq:7} we get
\begin{equation*}
  \E(I_t + R_t - B_t) = r+1,
\end{equation*}
from which
\begin{equation}
  \label{eq:6}
  \begin{split}
  \E(I_t)
  &= r+1 - \E(R_t) + \E(B_t)  \\
  & = r + 1 - \E(X(t-L, t]) + B(r) \\
  & = r + 1 - \theta + B(r) \\
  \end{split}
\end{equation}

Formulas to skip (in edition 3): 2.24, 2.25. 


\subsubsection{Simulation of the basestock inventory model}
\label{sec:simul-basest-invent}

Consider a periodic-time model so that $\IP_i$ is the inventory position at the end of period $i$. Then the sequence $\{\IP_i\}$ must satisfy the recursion:
\begin{equation}
  \label{eq:15}
  %\IP_i = (r+1-\IP_{i-1}-X_i)1\{\IP_{i-1} - X_i\leq r\}.
  \IP_i = r+1 = r+1 - X_i + X_i 1\{X_i>0\}.
\end{equation}
To see this, observe that under the basestock policy the inventory
position is always kept at level $r+1$, c.f., (\ref{eq:20}). Thus, if
the inventory position at the end of period $i-1$ minus the demand
$X_i$ during period $i$ is less than $r+1$, we need to reorder the
shortage. Since the shortage is precisely the demand during period
$i$, i.e., $X_i$, we order $X_i$. (Recall that $1\{X_i>0\} = 1$ if
$X_i>0$ and is $0$ otherwise.)

When the leadtime $L$ is one period or more, the replenishments do not arrive right away but $L$ periods later. The consequences of the inventory level are that
\begin{equation}
  \label{eq:21}
  \IL_i = \IL_{i-1} - X_i + X_{i-L})1\{X_{i-L}>0\}.
\end{equation}
Thus, what we ordered $L$ periods `ago', we receive `now'.

Once we carry out a simulation for $n$ periods, we can estimate the
performance measure. The average inventory level is
\begin{equation}
  \label{eq:22}
  I = \frac 1n \sum_{i=1}^n \IL_i1\{\IL_i \geq 0\},
\end{equation}
the average backlog  is
\begin{equation}
  B = - \frac 1n \sum_{i=1}^n \IL_i1\{\IL_i < 0\},
\end{equation}
because $\IL_i<0$ if there are backorders, recall~(\ref{eq:23}). 
The service level is 
\begin{equation}
  \label{eq:22}
  S = \frac 1n \sum_{i=1}^n \{\IL_i \geq 0\},
\end{equation}




%%% Local Variables:
%%% mode: latex
%%% TeX-master: "notes_all"
%%% End:
