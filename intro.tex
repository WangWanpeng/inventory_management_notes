
We organize and motivate course notes along a set of exercises. The aim of this approach is to motivate you to think and discuss on the fundamentals of inventory theory with your fellow students. We provide a set of solutions for all the exercises to prevent you from getting stuck. In case you identify interesting questions that we do not address and/or come across questions or solutions that are not clear, please let us know. 

We often formulate problems in terms of symbols (rather than numbers) as this is much clearer eventually. It will also help you implement the models that we are going to develop in a spreadsheet application, or in some more useful programming environment. We urge you to implement all solutions in a spreadsheet application. Once you know how to do it, you will see that it is fairly easy and provides enormous insight.

As a guide to solving the exercises; keep in mind that they are not meant to be easy (otherwise we could have asked you to check whether $11+22=33$). Thus, it is not a real issue or a ``failure'' if you cannot solve an exercise. It is a failure, however, not to \emph{attempt} to solve an exercise. Always think hard about an exercise before you check the solution. We expect that when you start on a new exercise once you have read and understood the solution of all previous exercises. 

We advise you to use the course notes along with the textbook Factory Physics 3rd Edition (FP). The notes will refer to relevant parts of your textbook whenever necessary. 
