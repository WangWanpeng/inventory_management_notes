\section{Introduction}
\label{sec:introduction}

We organize and motivate course notes along a set of exercises. The aim of this approach is to motivate you to think and discuss on the fundamentals of inventory theory with your fellow students. We provide a set of solutions for all the exercises to prevent you from getting stuck. In case you identify interesting questions that we do not address and/or come across questions or solutions that are not clear, please let us know. 

We often formulate problems in terms of symbols (rather than numbers) as this is much clearer eventually. It will also help you implement the models that we are going to develop in a spreadsheet application, or in some more useful programming environment. We urge you to implement all solutions in a spreadsheet application. Once you know how to do it, you will see that it is fairly easy and provides enormous insight.

As a guide to solving the exercises; keep in mind that they are not meant to be easy (otherwise we could have asked you to check whether $11+22=33$). Thus, it is not a real issue or a ``failure'' if you cannot solve an exercise. It is a failure, however, not to \emph{attempt} to solve an exercise. Always think hard about an exercise before you check the solution. We expect that  you only start with a new exercise after you have read and \emph{understood} the solutions of \emph{all} previous exercises. 

Sometimes we use code (Python) for numerical work. As some students
are interested in seeing how we actually carried out the computations
we sometimes include the code in these notes. However, the code is not
obligatory for the course.

Some of the exercises are a bit technical, or require some mathematical tools that are not directly necessary for the course. These exercises are marked with this symbol \faRocket. You do not have to study the solution of such exercises; however, you do have to be able to apply the results of such exercises.

So\nvf{to do} in the notes below we typically start with making a very simple model, a case description if you
will, and make it more complicated as we go along.  You are supposed
to follow the steps and implement them in a spreadsheet, make graphs, compute
KPIs, and so on, to test your understanding of the approach to problem
solving we want you to learn.

\nvf{We start with a single period inventory model, the newsvendor model, and then extend the horizon. Since our focus is on simulation, this makes sense. We first setup an environment to carry out simulations. Then we make analytical solutions and show how the results of the simulation lead to these analytical results.}
\clearpage
