\documentclass{article} 

\usepackage{fullpage}
\usepackage{ctable}
\usepackage{multirow}
\usepackage{amsmath,amssymb}
\usepackage{hyperref}
\usepackage{natbib}
\usepackage{xspace}

\usepackage{tocloft}
\setcounter{tocdepth}{2}
\setlength\cftbeforesecskip{2.5pt}

\usepackage{xcolor}
\hypersetup{
    colorlinks,
    linkcolor={red!50!black},
    citecolor={blue!50!black},
    urlcolor={blue!80!black}
}

\usepackage{enumitem}
\setlist{leftmargin=1cm,labelsep=10pt,style=sameline,topsep=0pt,itemsep=1pt,partopsep=10pt}
\setlist{nosep,after=\vspace{.1cm}}

\usepackage{soul}
\setstcolor{red}

\usepackage{sectsty}
\sectionfont{\bfseries\large}
\subsectionfont{\bfseries\normalsize}

\setlength{\parskip}{.1cm}
\renewcommand{\baselinestretch}{1.25}

\newcommand{\nestor}{\href{nestor.rug.nl}{\texttt{nestor}}\xspace}
\newcommand{\policy}{\href{http://www.rug.nl/feb/education/exchange/courseinformation/fraudeenplagiaat?lang=en}{\texttt{policy on fraud and plagiarism}}\xspace}
\usepackage{tocloft}



\title{\bf
	Course Manual \\ 
	Inventory Management \\
	{\normalsize 2017--2018}}

\date{\small updated on \today}
\author{\empty}

\begin{document}

\maketitle

\begin{table}[htbp]
    \begin{tabular}{ll}
    \textbf{Course Code}  	& EBM026A05 \\
    \textbf{Study Load} 	& 5 ECTs \\
    \textbf{Lecturers} 		& dr. O.A. Kilic (o.a.kilic@rug.nl) -- coordinator \\ 
    		  				& dr. N.D. van Foreest (n.d.van.foreest@rug.nl) \\
    \textbf{Secretary} 		& 5411.0634 (secr.operations.feb@rug.nl) \\
    \end{tabular}%
\end{table}%


\setcounter{tocdepth}{2}
\setlength\cftbeforesecskip{2.5pt}
{\small\tableofcontents}

\newpage

\section{Course Description and Objectives}

This course provides students with the necessary knowledge and skills to analyze, improve, design and manage actual inventory systems. Topics that will be discussed include demand forecasting, production/inventory and item characteristics, performance measures, inventory control policies, and policy improvement. In particular, applying inventory control models to improve the company performance will be important. Throughout the course, spreadsheet modeling skills are essential.

After successful completion of this course, the students will be able to (1) analyze inventory management practices, (2) apply inventory control models, and (3) design procedures to improve inventory management in practice.


\section{Teaching and Assessment}

The course is delivered through \textit{lectures}, \textit{tutorials}, and a \textit{computer practical}. Lectures provide a theoretical background and an overview of modeling inventory systems. Tutorials are feedback moments where students present their work and discuss their findings with fellow students and the lecturers. The computer practical is meant to assist students with their spreadsheet modeling skills. Besides the aforementioned organized contact moments, lecturers will have weekly office hours where students can walk-in without making an appointment for questions and feedback. 
 
The assessment is done through an \hyperref[sec:individual]{\textit{individual assignment}}, a \hyperref[sec:group]{\textit{group assignment}}, and a \hyperref[sec:exam]{\textit{written exam}}. The individual assignment assesses your spreadsheet modeling skills which are essential for the rest of the course. The group assignment is a full fledged project that is carried out by teams of four students. The aim is to select and apply inventory management methods to analyze and improve a practical problem faced by case a company. The exam is based on the theoretical background on inventory systems covered in the lectures. 

The timing and location of all contact moments and the deadlines of assignments as well as the timing and location of the exam and the resit can be found in the \hyperref[sec:agenda]{course agenda}.


\section{Entry Conditions}

The students expected to be familiar with (1) basic inventory control concepts; such as, economic order quantity, reorder point, order-up-to level, lead time demand, and safety stock -- which have already been covered in undergraduate courses as well as other master courses in the field of supply chain and technology and operations management, (2) basic statistics and stochastics, and (3) spreadsheet modeling.


\section{Overview}

The course is designed such that students gain the necessary insights and model building skills to make use of inventory management in their future career. To that end, we will discuss how to deal with real-life inventory systems. This typically involves four stages:
\begin{enumerate}
\item Demand analysis and forecasting
\item Inventory modeling
\item Policy selection
\item Policy evaluation and system improvement
\end{enumerate}

These stages will be addressed in the lectures -- over the period of 1st--6th weeks of the course. The individual assignment makes sure that students have the necessary spreadsheet skills to carry out the computational chores of these stages. The knowledge on the basic concepts is often not sufficient to improve on actual inventory management problems. This is due to several reasons. First, despite a lot of information is available in practical cases, the quality of information is not always as required. Second, many practical problems are too complex to be solved by using simple models described in textbooks, and new models need to be designed based on adaptations of existing ones as well as insights from the literature. Finally, the assumptions behind textbook models are often not valid, and they need to be checked. The group assignment will provide students with hands-on experience to deal with these issues. 


\section{Reading Material}

\begin{enumerate}
\item Factory Physics (Hopp and Spearman, 3rd edition).
\item Course notes: The notes are centered around the topics covered in the lectures, and they are organized by means of exercises to make students aware of the main trade-offs as well as all degrees of freedom that exists in inventory management. Besides, they also give students an idea of the variety of questions that will appear in the exam. The notes will gradually be made available on \href{nestor.rug.nl}{\texttt{nestor}}.
\item Thinkstats (Downey, 2nd Edition). It is a nice book that you can consult on statistics, and it is freely available.
\end{enumerate}


\section{Rules and Grading}

\subsection{Grading Scheme}

The overall course grade will be based on two assignments and a written exam. The grading scheme is as follows: 
\begin{enumerate}
\item The first assignment is the individual assignment ``Spreadsheet Modeling Skills''. Its grade is either 1 (very bad), 5.5 (barely ok) or 10 (very good). The weight of this assignment is 10\%.
\item The second assignment is the group assignment ``Inventory Management in Practice''. The assessment is made on the final report which will be graded on a 1--10 scale (rounded off to one decimal). Its weight is 50\%. 
\item The written exam involves open questions. It is graded on a 1--100 scale. The weight of the exam is 40\%. 
\end{enumerate}

The following requirements apply to pass the course: (1) the group assignment grade should at least be 5.5, and (2) the exam grade should at least be 5. It is possible to submit a repair for the group assignment. However, the maximum grade for a repaired assignment is 5.

\subsection{Rules}

The general course rules are as follows:
\begin{enumerate}
\item All assignments must be submitted by their deadlines to be graded. Late submissions are accepted only at the discretion of the course coordinator, and penalties may be imposed. 
\item All course information and announcements are posted on \nestor. We advise students are to check the course page regularly.
\item We will take fraud and plagiarism very seriously. We advise students to carefully read FEB's \policy.
\end{enumerate}

 
\section{Assignments and Exam}

\subsection{Individual Assignment: ``Spreadsheet Modeling Skills''}
\label{sec:individual}

The individual assignment assesses your spreadsheet modeling skills which are essential for the rest of the course -- especially the group assignment requires a high level of spreadsheet modeling skills. The focus of the assignment is on economic order quantity, inventory dynamics, demand distributions, and forecasting. The assignment will be made available on \href{nestor.rug.nl}{\texttt{nestor}}. We suggest students to start working on the assignment right away. If you need support on completing the assignment, you are welcome to one of the computer practicals that will be held in the 1st week of the course. 

The answers should be submitted in the 2nd week of the course and uploaded on \href{nestor.rug.nl}{\texttt{nestor}}. We encourage you to collaborate with other students for this assignment and compare your results. You should also test your work before submitting your assignment. The workload for this assignment strongly depends on your background. But it should not exceed 12 hours, as it would otherwise mean that your skills should improve quite dramatically. 

\subsection{Group Assignment: ``Inventory Management in Practice''}
\label{sec:group}

The group assignment is a full fledged project that is carried out by teams of four students. It is an important part of the course which requires students to select and apply inventory management methods to analyze and improve a practical problem faced by case a company. There are four different cases in total which are centered around different inventory management problems supported by real-life data. The cases are not easy to say the least. Also, they do not have a single correct solution. We therefore encourage students to have an open mind and be innovative. 

In the 1st week of the course, students will enroll to the group assignment as teams. Then, we assign a case to each team. The case descriptions and data as well as the assessment criteria for the assignment will be available on \nestor.

\paragraph{Structure} The group assignment will require data analysis, selecting adequate methods from literature, model building in spreadsheets, and problem solving. Teams will conduct these activities in four consecutive stages: 
\begin{enumerate}[leftmargin=*,label=Stage \arabic*:]
\item Demand analysis and forecasting: Establish a demand model and characterize the (distribution of the) demand during the replenishment lead time.
\item Inventory modeling: Build a mathematical model of the inventory system and analyze its behavior. Devise a cost model. Determine system objectives and quantify these in terms of KPI's.
\item Policy selection: Determine the decision variables by which the system can be controlled. Select a suitable inventory policy. Find appropriate parameter settings with the aim  to let the policy behave according to expectation. Implement the policy and assess its performance by means of simulation.
\item Policy evaluation and system improvement: Conduct robustness and sensitivity analyses. Assess the policy of choice against other policies in terms of KPI's and show that it indeed performs better. Seek for out-of-the-box ideas to improve the system even more.
\end{enumerate}

The process will be supported by lectures where we discuss theory and provide simple examples. Teams should apply the theory to their case and develop their own ideas on how to analyze and improve the inventory system.

All teams are expected to work on the stages mentioned above in four consecutive weeks -- over the period of 3rd--6th weeks of the course. During this period, teams will report their progress and receive feedback on one of the four stages each week. This will be done through \hyperref[par:presentation]{\textit{poster presentations}} and \hyperref[par:reflection]{\textit{reflection reports}}. The poster presentations will be made in tutorial sessions where each team will present and discuss the results they obtained in the corresponding stage of their project. The reflection report will provide a short summary of your progress. The posters along with a reflection report should be submitted on \nestor each week following the tutorial session. 

There will be one last feedback moment once tutorial sessions have been completed. Following the last tutorial session in the 6th week of the course, teams will put together a \textit{draft report} (besides their poster presentation and reflection report) which summarizes their models and findings regarding all four stages of their assignment. We will then pair each team with another. These teams will exchange their draft reports and provide each other with points for improvement. The lecture in the 7th week of the course will be reserved for discussions between groups on their feedback. 

The assignment will be graded based on your \hyperref[par:final]{\textit{final report}}. The report should be handed in on \nestor in the 7th week of the course.

\paragraph{Poster presentation}
\label{par:presentation}

The poster presentation should explain your ideas and results on specific stages of your assignment. The presentation should be limited to 4-5 minutes, and the remaining time is reserved for questions and discussion. We encourage teams to discuss their work with others and exchange ideas during the tutorial sessions. 

The important elements of the poster presentation are as follows: (i) make your question and objective explicit, (ii) explain your business problem well, otherwise your your will not be understood solution either. let figures tell your story and Prevent use of text and tables as much as possible, (iii) explain your approach well, (iv) share all the insights you gained, and (v) make next steps explicit.

We provide a poster format on \nestor for convenience. If you send your presentation two days prior the tutorial sessions before noon to \href{mailto:secr.operations.feb@rug.nl}{\texttt{secr.operations.feb@rug.nl}}, they will have it printed for you in color without charge. The prints will then be available in 5411.0631 or 5411.0634 (open until 4:30pm).

\paragraph{Reflection report}
\label{par:reflection}

Following each poster presentation, teams should reflect on what should be improved with respect to the corresponding stage of their assignment. The aim is to make the points of improvement explicit and assign tasks to individual group members. 

The reflection report should answer (in one or two sentences per item) the following: (1) which steps were taken in this week's stage? (2) what did you expect from these steps? (3) what actually happened? (4) what were the obstacles that prevented you from realizing your goals and what are you planning to do about those? (5) what are the things that need to be done next week? (6) which group members are responsible for these and when they will be done? 

\paragraph{Final report}
\label{par:final}

The final report should provide a comprehensive overview of your assignment project. Its content should include a concise case description and detailed overview of your approaches and results on all stages of the assignment. It should also provide conclusions and managerial recommendations. The report should not exceed 20 pages (including titlepage, references, and appendices) and 5000 words. The grading scheme of the final report will be available on \nestor.

\subsection{Exam}
\label{sec:exam}

The exam will cover the theoretical and computational background on inventory systems provided in the course. It will be based on the lectures, course notes, and relevant chapters and exercises of the course textbook Factory Physics. The exam is open book. That is, you will be allowed to use the textbook during the exam but nothing else. The exam will involve open questions with a computational nature. We expect you to excel in the basic models and methods as well as implementing them. The grading will therefore be result-driven and points can only be collected if the answer is correct. The course notes will provide a large variety of questions (together with their answers) that you can expect to see in the exam. We will also provide you with a mock exam. It will be made available on \nestor. 

We will post the answers to exam questions on \nestor immediately after the exam and release your grades in one week. There is a relatively short amount of time between the exam and the resit in this block. Therefore, we strongly advise you to start studying for the resit as early as possible if you believe that you have failed the exam.


\section{Course Agenda}
\label{sec:agenda}

The course agenda provides you with a complete overview of all the contact moments as well as the tasks that you should perform on a weekly basis. 

The course consists of lectures and tutorials. We expect students to attend all lectures. Also, we will have weekly office hours where you can walk-in without making an appointment. The scheduled office hours are 
\begin{enumerate}
\item O.A. Kilic -- Mon 1--2pm 5411.0657
\item N.D. van Foreest -- Mon 1--2pm 5411.0656
\end{enumerate}

\paragraph{Week 1} {\footnotesize (16th calendar week)}
\begin{description}[font=\normalfont\itshape,leftmargin=!,labelwidth=2cm]
\item[Tue 18/4] Lecture (9--11am, 5419.0013)
\item[Thu 20/4] Deadline (midnight): Enroll for the group assignment on \nestor. Teams should be comprised of four students. If there are less students in a team, we will manually assign team members to other teams. Enrollment will be open from the start of the course until the deadline. Note that students that are in the double degree program will be enrolled automatically by the coordinator. 
\item[Fri 21/04] Computer practical (9--11am 5415.0032 and 11am--1pm 5415.0042): You are welcome to one of the computer practicals if you need support on completing the individual assignment.
\item[Fri 21/04] We will assign each team to a case (one of the four cases) and a tutorial session (morning and afternoon), and inform you by email.
\end{description}

\paragraph{Week 2} {\footnotesize (17th calendar week)}
\begin{description}[font=\normalfont\itshape,leftmargin=!,labelwidth=2cm]
\item[Mon 24/4] Lecture (11am--1pm, 5419.0112)
\item[Wed 26/4] Deadline (midnight): Submit your individual assignment in an Excel file (xls or xlsx) on \nestor. 
\end{description}

\paragraph{Week 3} {\footnotesize(18th calendar week)}
\begin{description}[font=\normalfont\itshape,leftmargin=!,labelwidth=2cm]
\item[Mon 1/5] Lecture (11am--1pm, 5419.0112)
\item[Wed 3/5] Tutorial (9--11am and 3--5pm 5412.0040): Poster presentations on demand analysis and forecasting.
\item[Fri 5/5] Deadline (midnight): Submit your poster and reflection report (both in pdf) on \nestor. 
\end{description}

\paragraph{Week 4} {\footnotesize (19th calendar week)}
\begin{description}[font=\normalfont\itshape,leftmargin=!,labelwidth=2cm]
\item[Mon 8/5] Lecture (11am--1pm, 5419.0112)
\item[Wed 10/5] Tutorial (9--11am and 3--5pm 5412.0040): Poster presentations on inventory modeling.
\item[Fri 12/5] Deadline (midnight): Submit your poster and reflection report (both in pdf) on \nestor. 
\end{description}

\paragraph{Week 5} {\footnotesize(20th calendar week)}
\begin{description}[font=\normalfont\itshape,leftmargin=!,labelwidth=2cm]
\item[Mon 15/5] Lecture (11am--1pm, 5419.0112)
\item[Wed 17/5] Tutorial (9--11am and 3--5pm 5412.0040): Poster presentations on policy selection.
\item[Fri 19/5] Deadline (midnight): Submit your poster and reflection report (both in pdf) on \nestor. 
\end{description}

\paragraph{Week 6} {\footnotesize(22nd calendar week)}
\begin{description}[font=\normalfont\itshape,leftmargin=!,labelwidth=2cm]
\item[Mon 29/5] Lecture (11am--1pm, 5419.0112)
\item[Wed 31/5] Tutorial (9--11am and 3--5pm 5412.0040): Poster presentations on policy evaluation and system improvement.
\item[Thu 1/6] We will pair teams with one another and inform you by email. These teams will exchange their draft reports and provide each other with points for improvement. 
\item[Fri 2/6] Deadline (midnight): Submit your poster and reflection report (both in pdf) on \nestor, and send your draft report (in pdf) to the team that you were paired.
\end{description}

\paragraph{Week 7} {\footnotesize(23rd calendar week)}
\begin{description}[font=\normalfont\itshape,leftmargin=!,labelwidth=2cm]
\item[Tue 6/6] Lecture (9--11am, 5419.0013): This lecture is reserved for discussions between paired teams on their draft report feedback. 
\item[Fri 9/6] Deadline (midnight): Submit your group assignment final report (in pdf) on \nestor.
\end{description}

\paragraph{Post-Lecture Period}
\begin{description}[font=\normalfont\itshape,leftmargin=!,labelwidth=2cm]
\item[Wed 21/6] Exam (9--11am, ACLO Station Stationsplein 7-9 9726AE Groningen, Frascati Entrance)
\item[Wed 6/7] Resit (2--4pm, A. Jacobshal 01).
\end{description}

\end{document}


