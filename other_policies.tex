\section{Other Inventory Models}
\label{sec:other-invent-models}

\Opensolutionfile{ans}

In this section we will develop sets of recursions for inventory systems under several types of inventory policy, thereby allowing the analysis of these rules by means of simulation.

Under the $(s,S)$ policy we order up to $S$ when the inventory position is at or below level~$s$. 
\begin{exercise}
  Make a set of recursions by which we can simulate an $(s,S)$ policy.
  \begin{solution}
The $(s,S)$ policy works like this: 
\begin{align*}
  Q_i &= (S-\IP_{i-1}) \1{\IP_{i-1}\leq s} \\
  \IP_{i} &= \IP_{i-1} + Q_{i} - D_i, \\
  \IL_{i} &= \IL_{i-1} + Q_{i-L} - D_i.
\end{align*}
The starting values $\IP_0$ and $\IL_0$ can be arbitrary.
  \end{solution}
\end{exercise}

\begin{exercise}
  Compare this policy to the recursions for the basestock policy with reorder level~$r$ and the $(Q,r)$ policy. What is the relation between $s$ and $S$, on the hand, and $r$ on the other?
  \begin{solution}
    Only the rule for $Q_i$ is different. Under the $(s,S)$-rule we order up to $S$ once the inventory position becomes less than $s$. While the letters differ, $s$ and $r$ play the same role. Once the inventory position becomes less than $r$, or $s$, an order is triggered.
  \end{solution}
\end{exercise}

\begin{exercise}
  What values to choose for $s$ and $S$ to model a basestock policy with reorder level $r$ .
  \begin{solution}
    This follows right away from the rules for the order quantity $Q_i$. In particular, a basestock model with reorder level $r$ is an $(s,S)$ policy with $s=r$ and $S=r+1$. 
  \end{solution}
\end{exercise}

\begin{exercise}
  Suppose that in a period at most 1 demand can arrive. Explain that the $(Q,r)$ policy and the $(s,S)$ policy are the same by relating the policy parameters in the right way.
  \begin{solution}
    Set $s=r$ and $S=r+Q$. 
  \end{solution}
\end{exercise}

Under the $(T,S)$ policy we order every $T$ periods, and we order up to level $S$. 

\begin{exercise}
Formulate a set of recursions to  simulate a $(T,S)$ policy.
\begin{solution}
To set up the recursion we need a way to determine that the period index $i$ is a whole multiple of the cycle length $T$. One way to establish this is like this. Let $\text{mod}\{i, T\}$ be the fractional part of $i/T$. Then, if $\text{mod}\{i, T\}=0$, $i$ is whole multiple of $T$. 
 With this, the order quantity becomes
\begin{align*}
  Q_i &= (S-\IP_{i-1}) \1{\text{mod}\{i, T\} = 0};
\end{align*}
the recursions for $\IP_i$ and $\IL_i$ are the same as for the $(s,S)$ policy.
It is also possible to keep track of the time since the last order period by means of some counter. Then, place a new order when this counter is equal to $T$ and reset the counter to 0. This, however, requires an (unnecessary) extra state space variable, i.e., the counter.
\end{solution}
  
\end{exercise}

Finally, the $(T,Q)$ policy specifies to place an order of size $Q$ every $T$ periods. 

\begin{exercise}
What  recursions can be used to simulate a $(T,S)$ policy?
\begin{solution}
The $(T,Q)$ policy is characterized like this. Let $Q$ be a fixed quantity, then,
\begin{align*}
  Q_i &= Q \1{\text{mod}\{i, T\} = 0}.
\end{align*}
Again, the recursions for $\IP_i$ and $\IL_i$ are the same as for the $(s,S)$ policy.
\end{solution}
\end{exercise}

\begin{exercise}
  Explain that the $(T,Q)$ policy will perform very badly for systems with stochastic demand. (Hint, what is the arrival rate of replenished items?)
  \begin{solution}
The replenishment rate under the $(T,Q)$ policy is $Q/T$, i.e., every $T$ periods we order an amount $Q$. Thus, this rate is constant. When demand is stochastic, the inventory system will behave as a random walk without drift (in the best case). Such random walks show very, very wild behavior. 
  \end{solution}
\end{exercise}

\begin{exercise}
  Consider the extensions to continuous time of all the policies we considered. Show how these policies reduce to the EOQ policy when demand is constant. 
  \begin{solution}
Let $Q^*$ be the optimal order quantity in the EOQ model.  
\begin{itemize}
\item In the  $(Q,r)$ policy, take $r=0$ and $Q=Q^*$.
\item In the  $(s,S)$ policy, take $s=0$ and $S=Q^*$.
\item In the  $(T,S)$ policy, take $T=Q^*/D$ and $S=Q^*$.
\item In the  $(T,Q)$ policy, take $T=Q^*/D$ and $Q=Q^*$.
\end{itemize}
  \end{solution}
\end{exercise}

\begin{exercise}
  In what respect are these policies (un)-predictable in case of stochastic demand?
  \begin{solution}
\begin{itemize}
\item $(T,Q)$ policy: completely predictable, but terrible behavior, either the inventory or the backorders increases to infinity. 
\item $(T,S)$ policy: order sizes vary, but order moments are fixed.
\item $(Q,r)$ policy: order sizes fixed, but order moments are variable.
\item $(s,S)$ policy: order sizes and  order moments fixed. This policy can achieve the lowest average cost,  but you do not know in advance the order time or order size. 
\end{itemize}
  \end{solution}
\end{exercise}

\begin{exercise}
  Give an practical example when each of these inventory policies are useful, except for the super silly $(T,Q)$ policy.
  \begin{solution}
    \begin{itemize}
    \item $(Q,r)$: Consider the example of a specific type of rice (for instance) in a supermarket with mean daily demand  2 and  standard deviation 1. Often rice is received in batched of 10 packets, sealed with a plastic wrap, from the rice manufacturer. Hence, the replenishment order size is more or less fixed. Its easy to take $Q=10$. Since the on-hand rice level is continuously monitored via the casher systems, its simple to trigger an order when the inventory position is 4 or so. Since the leadtime (time from the distribution center to the supermarket is a typically day, or less), and $Q$ is significantly larger than the average daily demand, the inventory position and inventory level mostly coincide. An $(s,S)$ policy is not appropriate for the fact that the supermarket cannot control the size of the deliveries; its simply 10.
    \item $(T,S)$: replenishing gas stations. It is a very hard job to determine good (short) routes for all the trucks. Many constraints need to be taken into account such as the time windows in which it is allowed to visit the stations, driver schedules, and so on. Thus, one approach is to first fix all the routes. Then it is known when the tank stations will be visited. Once a truck arrives, it is a simple strategy to replenish the different gas types to a standard level. So, even though an $(s,S)$ policy might be deployed here (all gas levels are known at any moment in time), the routing constraints are such that stochastic cycle times are not practical. 

More generally, joint ordering systems are relatively easily controlled with $(T,S)$ policies. First determine suitable cycle times for the different SKUs. Then distribute the delivery periods reasonably so that the work load (either in time or in volume) over the periods is more or less even. Finally, determine reasonable replenishment levels.
    \item $(s,S)$: packaged meat, like bacon, in a supermarket. Suppliers of packaged meat deliver in single units, e.g. 100 gr of sealed bacon. Thus, there is no need for a fixed $Q$. The inventory levels are tracked real time, and there are daily deliveries. As long as the inventory level is suficiently high, there is no need to replenish (hence a basestock policy is not useful). In fact, as we have to deal with perishability here, it seems best to only replenish when there are very little items left. Then we put the newest  items under the older items, so that customers tend to pick the oldest items first. 
    \item Base stock. This is the typical policy in kanban system. For each item used, a replenisment card  is sent upstream. When the inventory is replenished up to level $r+1$, all cards are bound to an item, hence there are no request for production. 
    \end{itemize}
  \end{solution}
\end{exercise}

\begin{exercise}
An ATM (automated  teller machine, i.e., a money machine) is replenished every three days with `new' money by replacing the `old' container by a `new' full container. What shorthand would you use to describe this policy, and why?
\begin{solution}
This is a periodic-review basestock policy with no lead-time, as money is replenished up to the same level (full container) every time the container is replaced. This policy is also known as the $(T,S)$ policy where $T$ is the length of the regular replenishment intervals and $S$ is the order up-to level. 

%Another policy could be a periodic-review $(s,S)$-policy, if the container was being replaced only when the money in it was less than some level $s$. 
\end{solution}
\end{exercise}

\begin{comment}
  This makes a great case. Suppose we want to move from $(T,S)$ based policies to $(s,S)$ based policies. This must affect routing, routes become dynamic under the $(s,S)$ policy. But the $S$ can be made lower, perhaps, in the $(s,S)$ case, and the containers are less often replaced. It must save man hours. 
\end{comment}

\begin{exercise}[Continuation]
The bank wants to provide a high quality of service to its customers. To that end, the managerial team aims at organizing a replenishment schedule such that the probability of an ATM being out-of-money on an arbitrary day is below 1\%. Based on  historical data,  the daily demand for money at a particular ATM is well described by a normal distribution with $\mu=\$9\,700$ and $\sigma=\$3\,200$. The money container can hold \$90\,000. How frequent (in days) should this ATM be replenished to guarantee the service target?
\begin{solution}
  
Recall that if $D_1,\ldots,D_n$ be $n$ normally distributed random variables with mean and standard deviation values $(\mu_1,\sigma_1),\ldots,(\mu_n,\sigma_n)$ the sum $D[1,n]=D_1+\ldots+D_n$ is also normally distributed, but with mean and variance, respectively, equal to
\begin{align*}
\mu_2+\mu_2+\ldots+\mu_m, & &\sigma_1^2+\sigma_2^2+\ldots+\sigma_m^2.
\end{align*}
Since all days have the same distribution, we have that 
\begin{align*}
  \E{D[1,n]} &= n \mu, & \V{D[1,n]} = n \sigma^2
\end{align*}
if the ATM is replenished every $n$ days. 

The service level criterion specifies a non-stockout probability for each day. Because the container holds $\$90 000$, we need to make sure that the probability of $D$ being less than or equal to $90\,000$ is above 99\%, i.e., 
\begin{align*}
  \P{D[1,n] \leq 90\,000} \geq 0.99,
\end{align*}
and we need to find $n$ such that this is true.  

% For any replenishment interval $n$ this value can easily be obtained through the standard normal cumulative distribution function. In Excel, this function is readily available (i.e. {\sc\small NORMSDIST($z$)}). But here we will use a $z$-table instead. 

Since $D[1,n]$ is normally distributed with $n 9\,700$ and $\sqrt{n}3\,200$, we have that
\begin{equation*}
  \frac{D[1,n]-n9\,700}{\sqrt{n} 3\,200}
\end{equation*}
is a standard normal distribution. Let $z$ be such that $\Phi(z) = 0.99$; hence, from a table with $z$ values we get that $z = 2.575829$.

Then, as always, we need to find the largest $n$ such that
\begin{equation*}
g(n) :=  \frac{90\,000-9\,700 n}{3\,200 \sqrt{n}} \geq 2.575829.
\end{equation*}
In the table below we compute $g(n)$ for various values.
\begin{pycode}
from math import sqrt

def thres(n):
    return (90000-n*9700)/(sqrt(n)*3200)

\end{pycode}
\begin{center}
  \begin{tabular}{ll}
    $n$& $g(n)$ \\
1 & \py{thres(1)} \\
2 & \py{thres(2)} \\
3 & \py{thres(3)} \\
4 & \py{thres(4)} \\
5 & \py{thres(5)} \\
6 & \py{thres(6)} \\
7 & \py{thres(7)} \\
8 & \py{thres(8)}
  \end{tabular}
\end{center}
Clearly, we need to replenish every $n=7$ days.

% We start from $n=1$. We compute the mean $n\mu$, standard deviation $\sqrt{n}\sigma$, and corresponding $z=\frac{80000-n\mu}{\sqrt{n}\sigma}$. Then, we find the non-stockout probability $F(z)$ from the $z$-table. We keep on increasing $n$ and repeating this as long as $F(z)$ is above $0.99$. The results are reported below.
% \begin{center}
%     \vspace*{.25cm}
%     \begin{tabular}{ccc}
%     $n$ & $z$ & $F(z)$ \\
%     \toprule
%     1     & 25.09 & 1.0000 \\
%     2     & 15.60 & 1.0000 \\
%     3     & 10.99 & 1.0000 \\
%     4     & 8.00  & 1.0000 \\
%     5     & 5.80  & 1.0000 \\
%     6     & 4.06  & 1.0000 \\
%     7     & 2.61  & 0.9955 \\
%     8     & 1.37  & 0.9147 \\
%     \bottomrule
%     \end{tabular}%
%     \vspace*{.25cm}
% \end{center}
% It is clear that the ATM should be visited at least every 7 days to gurantee the prescribed service quality, as going for the 8th day reduce the non-stockout probability from \%95.55 to \%91.47 all at once.
\end{solution}
\end{exercise}

\begin{exercise}[Continuation]
  If we use a replenishment period of 3 days, how much money should the container contain to ensure a service level of $1\%$?
  \begin{solution}
We now need to find an amount $S$ such that
\begin{equation*}
\frac{S-3\mu}{\sqrt{3} \sigma} \geq z = 2.575829.
\end{equation*}
\begin{pyconsole}
from math import sqrt

mu = 9700
sigma = 3200
z = 2.575829
S = z*sqrt(3)*sigma + 3*mu
S
\end{pyconsole}

Taking into account that ATMs don't distribute coins, let's put $S=\$43\,400$ in the container.
  \end{solution}
\end{exercise}

\begin{comment}

We shall cover here the following policies

Possible exercises:
\begin{itemize}
\item an exercise on why $(T,Q)$ is silly
\end{itemize}
  
\end{comment}



\Closesolutionfile{ans}
\opt{solutionfiles}{
\subsection{Solutions}
\input{ans}
}

\clearpage
%%% Local Variables:
%%% mode: latex
%%% TeX-master: "inventory_notes"
%%% End:


