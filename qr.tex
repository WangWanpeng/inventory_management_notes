\section{(Q,r) Model}
\label{sec:q-r-model}

\Opensolutionfile{ans}

The $(Q,r)$ policy is a generalization of the basestock policy to deal with systems that involve ordering costs besided inventory and backlogging costs. 

\subsection{Simulation}

We develop a number of recursions by which we  simulate an inventory system under a $(Q,r)$-rule. The reader should notice that the rules share a similarity to the those of the base-stock policy, but there are subtle differences. 

First we deal with the dynamics of the inventory position $\IP_i$ at the end of period $i$. When $\IP_{i-1}\leq r$, we place at the start of period $i$ a number $n_i$ of batches, each of size $Q$, such that $\IP_{i-1} + n_i Q > r$. Then, letting  $n_i=0$ if $\IP_i \geq r+1$, define the order quantity as 
\begin{equation}\label{eq:20}
Q_i = n_i Q \1{\IP_{i-1}\leq r}
\end{equation}

\begin{exercise}
  Show that the basestock policy is a $(Q,r)$ policy with $Q=1$. 
\end{exercise}


Combining the expression for $Q_i$ with $\IP_{i-1}$ we arrive at the recursion
\begin{equation}\label{eq:6}
  \IP_{i} = \IP_{i-1} + Q_{i} - D_i.
\end{equation}


\begin{exercise}
  What is the largest value of $n_i$ if $Q$ is larger than the largest possible demand, i.e,. $Q\geq D_i$ for all $i$?
  \begin{solution}
    In such case $n_i\leq 1$: suppose $\IP_{i-1} > r$ and $\IP_i \leq r$, then 
    \begin{equation*}
\IP_i > r-D_i \geq r - Q.
    \end{equation*}
Thus, by ordering $Q$, $\IP_{i+1} > r$ again.
  \end{solution}
\end{exercise}


\begin{exercise}
The inventory position at the start of period $i$ is equal to $\IP_{i-1} + Q_i$.   Explain that 
\begin{equation*}
  r+1 \leq \IP_{i-1}+Q_i \leq r + Q.
\end{equation*}
\begin{solution}
  If $\IP_{i-1} \leq r$, we order the number $n_i$ of batches $Q$ such that $r+1\leq \IP_{i-1} + n_i Q \leq r+Q$. 
\end{solution}
\end{exercise}

Similar to the basestock model, replenishments arrive $L$ period later. Hence, the inventory level satisfies~\eqref{eq:b5}, i.e., 
\begin{equation}\label{eq:1}
  \IL_{i} = \IL_{i-1} + Q_{i-L} - D_i.
\end{equation}

\begin{exercise}
How do the recursions for the $(Q,r)$-policy differ from those of the basestock policy?
  \begin{solution} Only the ordering triggering rules for $Q_i$ differ. Compare~\eqref{eq:11} and \eqref{eq:20}.
  \end{solution}
\end{exercise}


It can be seen from the above that

\begin{exercise}
Finally, show that
  \begin{equation}\label{eq:7}
    \IL_i = \IP_i - \sum_{j=i-L+1}^i Q_j.
  \end{equation}
Thus, noting that $\sum_{j=i-L+1}^i Q_j$ is the amount of outstanding orders in period $i$, we see that $\IL_i$ is the inventory position $\IP_i$ at the end of the period minus all outstanding orders.
\begin{solution}
  If we   subtract~\eqref{eq:21} from \eqref{eq:8} we get 
  \begin{align*}
    \IP_i - \IL_i 
&= \IP_{i-1} + Q_{i} - \IL_{i-1} - Q_{i-L} \\
&= \IP_{i-2} - \IL_{i-2} + \sum_{j=i-1}^i Q_{j} - \sum_{j=i-L-1}^{i-L} Q_{j} \\
&= \IP_{0} - \IL_{0} + \sum_{j=1}^i Q_{j} - \sum_{j=1}^{i-L} Q_{j} &\text{as }Q_j=0, j\leq 0\\
&= \sum_{j=1}^i Q_{j} - \sum_{j=1}^{i-L} Q_{j} &\text{as }\IP_0 = \IL_0\\
&= \sum_{j=i-L+1}^i Q_{j}.
  \end{align*}
\end{solution}
\end{exercise}



It can be seen from the above that
  \begin{equation}\label{eq:7}
    \IL_i = \IP_i - \sum_{j=i-L+1}^i Q_j.
  \end{equation}
Thus, noting that $\sum_{j=i-L+1}^i Q_j$ is the amount of outstanding orders in period $i$, we see that $\IL_i$ is the inventory position $\IP_i$ at the end of the period minus all outstanding orders.

\begin{exercise}
  Subtract \eqref{eq:6} and \eqref{eq:1} to show \eqref{eq:7}.
\begin{solution}
  If we   subtract \eqref{eq:6} from \eqref{eq:1} we get 
  \begin{align*}
    \IP_i - \IL_i 
&= \IP_{i-1} + Q_{i} - \IL_{i-1} - Q_{i-L} \\
&= \IP_{i-2} - \IL_{i-2} + \sum_{j=i-1}^i Q_{j} - \sum_{j=i-L-1}^{i-L} Q_{j} \\
&= \IP_{0} - \IL_{0} + \sum_{j=1}^i Q_{j} - \sum_{j=1}^{i-L} Q_{j} &\text{as }Q_j=0, j\leq 0\\
&= \sum_{j=1}^i Q_{j} - \sum_{j=1}^{i-L} Q_{j} &\text{as }\IP_0 = \IL_0\\
&= \sum_{j=i-L+1}^i Q_{j}.
  \end{align*}
\end{solution}
\end{exercise}


\begin{exercise}
  Show how~\eqref{eq:7} reduces to \eqref{eq:b2} for the basestock model.
  \begin{solution}
For the basestock policy
    \begin{align*}
      \IL_i &= r+1 - D[i-L, i] & \text{by \eqref{eq:b2}} \\
&= r+1 -D_i - D[i-L, i-1] \\
&= \IP_i - D[i-L, i-1] & \text{ by \eqref{eq:9}}.
    \end{align*}
But, also,     for the basestock policy we have, by~\eqref{eq:22b}, that 
\begin{align*}
\sum_{i=i-L+1}^i Q_j 
&= \sum_{i=i-L+1}^i D_{j-1} \\ 
&= \sum_{i=i-L}^{i-1} D_{j} \\ 
&=D[i-L, i-1].
\end{align*}
  \end{solution}
\end{exercise}

\begin{exercise}
  Explain that if $Q>1$, in general,
  \begin{align*}
    \P{Q_i= k} \neq \P{D_i = k}.
  \end{align*}
Hence, the distribution of $\sum_{j=i-L}^i Q_j$ is not the same as $D[i-L, i]$. 
\begin{solution}
It can happen that during a period just one customer arrives, hence  $\P{D_i=1}>0$. However, $\P{Q_i=1}=0$ since we never order in single units. 
\end{solution}
\end{exercise}

As we will see, for the  analysis  of the $(Q,r)$-policy, \eqref{eq:7} is much less useful than~\eqref{eq:15}. 

The performance measures for the $(Q,r)$-model are identical to those of the basestock model. The cost should include also the ordering costs. 

\begin{exercise}
  Explain that the average costs becomes
  \begin{equation*}
    \frac A  n \sum_{i=1}^n \1{Q_i > 0} + h I + b B.
  \end{equation*}
  \begin{solution}
    The second and third term correspond to the average holding and backlogging cost. For the first, observe that $\sum_{i=1}^n \1{Q_i>0}$ counts the number of times an order is placed during the simulation with a duration of $n$  periods. 
  \end{solution}
\end{exercise}

\begin{comment}
If you were to plot the inventory positions as a set of points, see Figure~\ref{fig:qr_demand} then
\begin{itemize}
\item $\IP_t$ becomes  the point $(t-1+\epsilon, \IP_t)$, 
\item $IP_t'$ becomes the point $(t-\epsilon, IP_t')$
\item $D_t$  becomes the point $(t-1/2, D_t)$
\item $Q_t$  becomes the point $(t, Q_t)$.
\end{itemize}

\begin{figure}[htbp]
  \centering
  \begin{tabular}[h]{cc}
\input{progs/qr_start_end_figures}\\
\input{progs/qr_level_figures}\\
  \end{tabular}
  \caption{Upper panel: Behavior of $\IP_t$ and $\IP_t'$ as functions of time. Lower panel: a graph of the inventory level $\IL_t$. Observe in the lower panel that the replenishments arrive $L=3$ periods later.}
\label{fig:qr_demand}
\end{figure}
\end{comment}

\begin{comment}
  \begin{enumerate}
  \item   Exercises on systems with loss
  \item graphical `proof' of insensitivity of cost on Q and r around the minimum.
  \end{enumerate}
\end{comment}

\subsection{Analytic Results}

For  the ready rate, i.e., $S_r(Q,r) = \P{\IL_i > 0}$ as a function of the policy parameters $Q$ and $r$ we find that
\begin{align}\label{eq:5}
   S(Q,r) = \frac1Q \sum_{k=r}^{r+Q-1} S_r(k) = \frac1Q \sum_{k=r}^{r+Q-1} F(k) 
\end{align}
where $S_r(i)=F(i)$ is the ready rate  of the basestock model with reorder
level $k$, and $F(k)$ is given by~\eqref{eq:16}. 

\begin{exercise}[\faRocket]
Derive
% (This is a technical exercise, you do not have to solve it if you are not interested in probability theory.)
\begin{solution}

\begin{align*}
  \P{\IL_i > 0} 
&= \P{\IP_{i-L} - D[i-L+1, i] >0} \\
&= \sum_{k}\P{\IP_{i-L} - D[i-L+1, i] >0\given \ldots} \\
\end{align*}
\end{solution}
\end{exercise}

\begin{exercise}
  What result do you get if the order quantity $Q=1$? 
  \begin{solution}
\begin{align*}
   S_r(1,r) = \frac11 \sum_{i=r}^{r+1-1} F(i) = S_r(r).
\end{align*}
This is the same formula as found for the basestock model.
  \end{solution}
\end{exercise}

\begin{exercise}
  For the demand of Exercise~\ref{q:basestock}, compute $S_r(2,1)$, i.e., $S_r(Q, r)$ for $Q=2$ and $r=1$. 

\begin{solution}
With $Q=2$ and $r=1$,
  \begin{align*}
      S(2,1)
&= \frac{1}Q\sum_{k=r}^{r+Q-1} F(k) \\
&= \frac{1}2\sum_{k=1}^{1+2-1} F(k) \\
&=  \frac 12 (F(1) + F(2) ) \\
&= \frac12\left(\frac{11}{30} + \frac{37}{60}\right) \\
&=\frac{59}{120}.
  \end{align*}

Here is the python code to compute the numbers. You don't have to memorize the code, but implementing in code helps to check the numbers.
  \begin{pyconsole}
from fractions import Fraction
p = [ Fraction(1,6),
      Fraction(1,5),
      Fraction(1,4),
      Fraction(1,8),
      Fraction(11,120),
      Fraction(1,6)
      ]

F = np.cumsum(p)
F

def S(Q,r):
    res = sum(F[i] for i in range(r, r+Q))
    return res/Q

S(2,1)
  \end{pyconsole}
\end{solution}
\end{exercise}



%\lstinputlisting[language=Python]{progs/basestock.py}

\begin{exercise}[\faRocket]
Show that 
\begin{equation}\label{eq:18}
   S(Q,r) = 1- \frac1Q [B(r-1) - B(r+Q-1)],
\end{equation}
where $B(r)$ is the expected number of backorders of the basestock model, i.e., it is given by \eqref{eq:12}. 
\begin{solution}
tbd
  
It appears that Eq. 2.70 of FP, edition 3, is off by one. To see
this, we prove that \eqref{eq:1} is indeed the same as \eqref{eq:5}. It
follows from \eqref{eq:1} and \eqref{eq:5} that
\begin{align*}
   B(r-1) - B(r+Q-1) 
   &= Q - Q S(Q,r) \\
   &= Q - \sum_{i=r}^{r+Q-1} S(i) \\
   &= \sum_{i=r}^{r+Q-1}(1- S(i)) \\
   &= \sum_{i=r}^{r+Q-1}(1- G(i)),
\end{align*}
since $\sum_{i=r}^{r+Q-1} 1 = Q$. From \eqref{eq:13} we see that $S(i)
= G(i)$. With \eqref{eq:18} this becomes
\begin{align*}
    B(r-1) - B(r+Q-1)
    &= \sum_{i=r}^{r+Q-1}(1- G(i))\\
    &= \sum_{i=r}^{r+Q-1} \bar G(i)\\
    &= \sum_{i=r}^{\infty} \bar G(i) -\sum_{i=r+Q}^\infty \bar G(i).
\end{align*}
From \eqref{eq:19} it follows that $B(r-1)=\sum_{i=r}^{\infty} \bar
G(i)$, and likewise for $B(r-1+Q)$. We are done.
\end{solution}
\end{exercise}

\begin{exercise}
  Check that the result of~\eqref{eq:5} and \eqref{eq:18} agree for the data of Exercise~\ref{q:basestock}. 
  \begin{solution}
Here is the code and some intermediate numbers so that you can check your computations.

    \begin{pyconsole}
def B(r):
    return sum((i-r-1)*p[i] for i in range(r+1, len(p)))
   
r = 1
Q = 2
   
B(r-1)
B(r+Q-1)
1-(B(r-1)-B(r+Q-1))/Q
    \end{pyconsole}
  \end{solution}
\end{exercise}

Conceptually, \eqref{eq:5} is more important than \eqref{eq:18}; however, \eqref{eq:18}  provides for a nice approximation.

\begin{exercise}
tbd:  Make an exercise to discuss this claim.
\end{exercise}

\begin{exercise}
Show that the expected number of back-orders is  given by
\begin{equation}
  \label{eq:14}
   B(Q,r) = \frac1Q \sum_{k=r}^{r+Q-1} B(k),
\end{equation}
where $B(i)$ is defined in \eqref{eq:12}. (This is technical)
\begin{solution}
tbd.
\end{solution}
\end{exercise}


\begin{exercise}
  Take the data from Exercise~\ref{q:basestock}. Suppose $r=1$ and $Q=2$. What is $B(Q,r)$?
\begin{solution}
Use~\eqref{eq:14} and the results of Exercise~\ref{q:basestock_B}
  \begin{align*}
      B(2,1)
&= \frac1Q \sum_{k=r}^{r+Q-1} B(k) \\
&= \frac12 \sum_{k=1}^{2} B(k) \\
&= \frac12 (B(1) + B(2)) \\
&= \frac12 \left(\frac{97}{120} + \frac{17}{40}\right) = \frac{37}{60}.
\end{align*}

\begin{pyconsole}
def BB(Q,r):
    res = sum(B(k) for k in range(r, r+Q))
    return res/Q

BB(2,1)
\end{pyconsole}
\end{solution}
\end{exercise}


\begin{exercise}[\faRocket]
Show that  the expected inventory level satifies
\begin{align}\label{eq:2}
   I(Q,r)
   = \frac1Q\sum_{k=r}^{r+Q-1} I(k)
   = \frac{Q+1}2 + r - \theta + B(Q,r), 
\end{align}
where $I(k)$ is the average inventory level of the basestock model with reorder level $k$, i.e.,  \eqref{eq:14}.
\begin{solution}
tbd

\begin{align*}
   I(Q,r)
   &= \frac1Q\sum_{i=r}^{r+Q-1} I(i) \\
   &= \frac1Q\sum_{i=r}^{r+Q-1} (i+1 - \theta + B(r)) \\
   &= \frac1Q\sum_{i=r}^{r+Q-1} (i + 1)  - \theta + \frac1Q\sum_{i=r}^{r+Q-1} B(r) \\
   &= \frac{Q+1}2 + r - \theta + B(Q,r), 
\end{align*}

\end{solution}
\end{exercise}




\begin{exercise}
  Take the data from Exercise~\ref{q:basestock}. Suppose $r=1$ and $Q=2$. What is $I(Q,r)$?

\begin{solution}
  Use~\eqref{eq:2} and the previous exercise and the result of Exercise~\ref{q:basestock}.

  \begin{equation*}
    I(2,1)  = \frac{2+1}2 + r - \frac{91}{40}+ \frac{37}{60} = \frac{101}{120}.
  \end{equation*}

Lets check it with the computer.
\begin{pyconsole}
theta = sum(i*p[i] for i in range(6))
theta

def I(Q,r):
    return Fraction(Q+1,2) + r - theta + BB(Q,r)

I(2,1)
\end{pyconsole}
\end{solution}
\end{exercise}

Formulas to skip (in edition 3): 2.38, 2.41, 2.42, 2.43.

\Closesolutionfile{ans}
\opt{solutionfiles}{
\subsection{Solutions}
\input{ans}
}


\clearpage
%%% Local Variables:
%%% mode: latex
%%% TeX-master: "inventory_notes"
%%% End:



