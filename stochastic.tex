%\textit{Recommended reading:} TBD
%
%
%\subsection{From Forecasts to Demand Distributions}
%
%\begin{question}
%What is the added value of considering stochastic demands? What if we simply use our forecast for planning?
%\end{question}
%
%\begin{solution}
%Let's begin with this: 'Forecasts are always wrong!' and 'Forecast errors are costly!'. Therefore, our models should acknowledge the fact that forecasts can be wrong. Furthermore, some forecast errors are more costly than others, e.g. over-forecast vs under-forecast. If the forecast is larger than the demand, then we have an overage cost (procurement in advance, holding, obsolescence); and if forecast is lower than the demand then we have a shortage cost (backorder cost, not meeting service quality target, loss of revenues). Often the shortage cost beat the overage cost. As a result of this, if real demand is distributed around some mean value, you order more than the mean. This is often referred to as the 'safety stock'. 
%
%To sum up, we need to integrate not only the the forecast but also the extent and the direction of the forecast error into our planning, as doing otherwise might have financial consequences. 
%\end{solution}
%
%\begin{question}
%I am confused. I have thought that we were forecasting to predict the demand. What forecast errors have to do with that?
%\end{question}
%
%\begin{solution}
%To answer this, we need to better understand what a forecast stands for. There is no perfect forecast as demand is inherently uncertain. Therefore, it is 'very normal' to have forecast errors. The question is whether your forecast errors match the inherent uncertainty in demand. That is, whether your forecast errors are originated from the demand uncertainty or they are a result of using a bad forecasting method. If it is the former, then you can safely assume that your point forecast is the average demand and your forecast errors are variations of demand from its average. If it is the latter, then you should do your best to devise a better forecasting method.
%
%Let us illustrate this with a simple example. Assume that demand is either 0 or 1, and the inherent uncertainty of the demand comes from a coin toss. That is, demand is 1 if it is heads and it is 0 if it is tails. As a planner; you do not know that demand indeed follows a coin toss, yet you need to forecast the demand. It is easy to see that there is no forecasting method you can use to predict the demand with certainty. In fact, the 'best' you can expect from a forecasting method is to tell you that the probability of having a demand of 0 units and 1 units are both 50\%. Here, if your forecast model provides a forecast error of 0.5 (almost) half of the time and -0.5 in the other half, then you are already there. Despite the sizeable amount of forecast errors, your forecasting method is just fine. On the other hand, if your forecast errors are, for instance, increasing or decreasing over time or have a tendency to be higher every two periods, then there is something wrong with your forecast method that should be repaired.
%
%Back to the original question; forecasts help us to predict average demand and it's variability from this average. Therefore they provide us the grounds for managing inventories. 
%\end{solution}
%
%\begin{question}
%How can we measure or assess the demand uncertainty?
%\end{question}
%
%\begin{solution}
%The answer can be found in the 'probability theory'. We can capture the uncertainty in demand through its probability distribution.
%\end{solution}
%
%\begin{question}
%What is a demand distribution?
%\end{question}
%
%\begin{solution}
%We should start with defining demand (over a pre-specified interval of time) as a random variable. Let this variable be $D$. Then, the (cumulative) distribution function of $D$ is given by $F(x)=\pr{D\leq x}$. That is, the probability that demand being less than or equal to some constant $x$. 
%\end{solution}
%
%\begin{question}
%How can we derive a demand distribution using forecasts? 
%\end{question}
%
%\begin{solution}
%There is no theoretically supported procedure for this purpose. A common approach is to approximate the variation of the demand with the variation of the forecast. To that end, first you need to make sure that the forecast indeed captures the trend and/or seasonality in the demand. This is the case if your forecast errors are relatively unbiased and do not follow a pattern over time. Then, you can approximate the mean demand with your point forecast and its deviation with the forecast errors. For instance; if your forecast errors are normally distributed over time, then you can assume that demand is normally distributed. Normal distribution is characterized by two parameters: mean and standard deviation. Then you can set the mean as your forecast and let the standard deviation be the square root of the mean squared error of your forecast. If your forecast errors are not normally distributed, then you can try and fit them to another distribution and/or use the empirical distribution (i.e. the distribution that you have observed). 
%
%Note that forecasting itself can be quite involved in many cases; for instance, where demand is auto-correlated and/or influenced by parameters that are not a part of your data. In such cases, standard time-series forecasting methods may not provide good forecasts at all. However, these issues are beyond the scope of this course. 
%\end{solution}
%
%\begin{question}
%What is the difference between the demand distribution and the lead time demand distribution?
%\end{question}
%
%\begin{solution}
%We construct demand distributions for demand over a given interval of time. If that interval is the lead time, then we simply have the lead time demand distribution. There are two possible ways of approaching the lead time demand distribution. First, you can start with forecasting the demand over a period and derive its distribution as usual and then use this distribution to construct the lead time demand distribution. For instance, if demand over a period is normally distributed with mean $\mu$ and standard deviation $\sigma$, then demand over $n$ periods (assuming that the demand in these periods follow the same distribution) is also normally distributed with mean $n\cdot\mu$ and standard deviation $\sqrt{n}\cdot \sigma$ (this property holds only for normal distribution, and the distribution). Second, you can directly start by forecasting the lead time demand rather than the period demand and derive its distribution.
%\end{solution}
%
%\begin{question}
%It appears that normal distribution comes in handy. How can you check whether a data set is normally distributed?
%\end{question}
%
%\begin{solution}
%There are many ways. There are very practical visual approaches, such as plotting the data as a histogram and checking against the normal distribution curve and making use of a normal distribution QQ-plot. Also, there are well-known statistical tests such as Shapiro-Wilk, Kolmogorov-Smirnov, and Anderson-Darling.
%\end{solution}
%
%\begin{question}
%Are there other distributions that are commonly used in inventory control, besides normal distribution?
%\end{question}
%
%\begin{solution}
%Normal distribution often does not work with slow movers, i.e. products with infrequent demands. For instance, if a product is sold only a few times over the lead time, it would not make sense to assume demand is normally distributed. In such cases Poisson distribution can be an option. 
%
%The preference towards normal and Poisson distributions is not only due to their 'nice' mathematical characteristics but also their power in capturing real-life demand data.
%\end{solution}
%
%

\subsection{The Newsvendor Model}

\pgfmathdeclarefunction{StandNormPdf}{1}{\pgfmathparse{1.0 / (sqrt(2 * pi)) * exp(-0.5 * (#1)^2)}}
\pgfmathdeclarefunction{StandNormCdf}{1}{\pgfmathparse{#1 > 0 ? (1.0 - 1.0 / sqrt(2 * (3.141592653589793238462643)) * exp(-(abs(#1)) * (abs(#1)) / 2) * ((0.31938153) * (1.0 / (1.0 + 0.2316419 * abs(#1))) + (-0.356563782) * (1.0 / (1.0 + 0.2316419 * abs(#1))) * (1.0 / (1.0 + 0.2316419 * abs(#1))) + (1.781477937) * pow((1.0 / (1.0 + 0.2316419 * abs(#1))),3) + (-1.821255978) * pow((1.0 / (1.0 + 0.2316419 * abs(#1))),4) + (1.330274429) * pow((1.0 / (1.0 + 0.2316419 * abs(#1))),5))) : (1 - (1.0 - 1.0 / sqrt(2 * (3.141592653589793238462643)) * exp(-(abs(#1)) * (abs(#1)) / 2) * ((0.31938153) * (1.0 / (1.0 + 0.2316419 * abs(#1))) + (-0.356563782) * (1.0 / (1.0 + 0.2316419 * abs(#1))) * (1.0 / (1.0 + 0.2316419 * abs(#1))) + (1.781477937) * pow((1.0 / (1.0 + 0.2316419 * abs(#1))),3) + (-1.821255978) * pow((1.0 / (1.0 + 0.2316419 * abs(#1))),4) + (1.330274429) * pow((1.0 / (1.0 + 0.2316419 * abs(#1))),5))))}}
\pgfmathdeclarefunction{StandNormLoss}{1}{\pgfmathparse{StandNormPdf(#1) - #1 * (1.0 - StandNormCdf(#1))}}
\pgfmathdeclarefunction{StandNormInv}{1}{\pgfmathparse{#1 < 0.02425 ? ((((((- 7.784894002430293e-03 * sqrt(-2 * ln(#1)) - 3.223964580411365e-01) * sqrt(-2 * ln(#1)) - 2.400758277161838e+00) * sqrt(-2 * ln(#1)) - 2.549732539343734e+00) * sqrt(-2 * ln(#1)) + 4.374664141464968e+00) * sqrt(-2*ln(#1)) + 2.938163982698783e+00) / ((((7.784695709041462e-03 * sqrt(-2 * ln(#1)) + 3.224671290700398e-01) * sqrt(-2 * ln(#1)) + 2.445134137142996e+00) * sqrt(-2 * ln(#1)) + 3.754408661907416e+00) * sqrt(-2 * ln(#1)) + 1)) : ((1 - 0.02425) < #1 ? (-(((((-7.784894002430293e-03 * sqrt(-2 * ln(1 - #1)) - 3.223964580411365e-01) * sqrt(-2 * ln(1 - #1)) - 2.400758277161838e+00) * sqrt(-2 * ln(1 - #1)) - 2.549732539343734e+00) * sqrt(-2 * ln(1 - #1)) + 4.374664141464968e+00) * sqrt(-2 * ln(1 - #1)) + 2.938163982698783e+00) / ((((7.784695709041462e-03 * sqrt(-2 * ln(1 - #1)) + 3.224671290700398e-01) * sqrt(-2 * ln(1 - #1)) + 2.445134137142996e+00) * sqrt(-2 * ln(1 - #1)) + 3.754408661907416e+00) * sqrt(-2 * ln(1 - #1)) + 1)) : ((((((-3.969683028665376e+01 * (#1 - 0.5)^2 + 2.209460984245205e+02) * (#1 - 0.5)^2 - 2.759285104469687e+02) * (#1 - 0.5)^2 + 1.383577518672690e+02) * (#1 - 0.5)^2 - 3.066479806614716e+01) * (#1 - 0.5)^2 + 2.506628277459239e+00) * (#1 - 0.5) / (((((-5.447609879822406e+01 * (#1 - 0.5)^2 + 1.615858368580409e+02) * (#1 - 0.5)^2 - 1.556989798598866e+02)*(#1 - 0.5)^2 + 6.680131188771972e+01)*(#1 - 0.5)^2 - 1.328068155288572e+01) * (#1 - 0.5)^2 + 1)))}}
\pgfmathdeclarefunction{Ldeter}{4}{\pgfmathparse{#3 * max(0,#1 - #2) + #4 * max(0,#2 - #1)}}
\pgfmathdeclarefunction{Lstoch}{5}{\pgfmathparse{#4 * (#1 - #2) + (#4 + #5) * #3 * StandNormLoss((#1 - #2) / #3)}}
\pgfmathdeclarefunction{Qopt}{4}{\pgfmathparse{#1 + #2 * StandNormInv(#4 / (#3 + #4))}}
\pgfmathdeclarefunction{Lopt}{4}{\pgfmathparse{(#3 + #4) * #2 * StandNormPdf(StandNormInv(#4 / (#3 + #4)))}}


The most well-known stochastic inventory model is the 'newsvendor model'. The following exercises will be devoted to this important model which clearly illustrates the effect of demand uncertainty in inventory decisions. 

The inventory control problem behind the newsvendor model can be explained as follows. Let us consider a newsvendor selling newspapers. The product here is perishable as newspapers does not have much of a value the day after they are printed. Therefore, the newsvendor plans his operations on a daily basis. There are a few of parameters that play a role: 
\begin{align*}
  c &= \text{Buying price of one item,} \\
  p &= \text{Selling  price of one item,} \\
  s &= \text{Salvage (end) value of one item}, \\
  h &= \text{Overage cost of one item ($c_o$ in Factory Physics)},\\
  b &= \text{Shortage cost of one item ($c_s$ in Factory Physics)}.
\end{align*}
 The newsvendor buys newspapers early in the morning before observing the demand, yet she does know the distribution of the daily demand. The problem of the newsvendor is to match her supply to the demand on a daily basis while maximizing his expected profit.


\begin{question}
How costs and price parameters should relate to each other so that the newsvendor problem to makes sense?
\end{question}

\begin{solution}
We must have $p>c>s$. That is because; If $p\leq c$ we would not buy anything as there would be no way of making profit, and if $c<s$ we would buy as much as possible as it would lead to more profit. 
\end{solution}

Let us assume that the newsvendor buys $Q$ items and demand turns out to be $d$ items. Keep in mind that here both $Q$ and $d$ are scalars, so there is no uncertainty. 

\begin{question}
Write an expression for the number of items sold.
\end{question}

\begin{solution}
\begin{align*}
\min\{Q,d\}
\end{align*}
\end{solution}

\begin{question}
Write an expression for the number of items salvaged.
\end{question}

\begin{solution}
\begin{align*}
Q-\min\{Q,d\}
\end{align*}
\end{solution}

\begin{question}
Write an expression for the newsvendor's profit $P(Q)$.
\end{question}

\begin{solution}
\begin{equation*}
  \begin{split}
P(Q) 
&= -cQ + p\min\{Q,d\} + s(Q-\min\{Q,d\}) \\
&=  -cQ + p \min\{Q,d\} + sQ - s \min\{Q,d\} \\
&= (s-c)Q + (p-s) \min\{Q,d\}.
  \end{split}
\end{equation*}
\end{solution}

The explanation at the beginning of this part defined the newsvendor problem as a 'profit maximization' problem. For the sake of brevity, we want to redefine it as a 'cost minimization' model. To that end, we can introduce two types of costs: unit overage cost $h$ (i.e. the cost of having an item on hand at the end of the day because it was not sold), and unit shortage cost $b$ (i.e. the cost of having a unit of unsatisfied demand due to stock-out). The next exercise relate to these cost components.

\begin{question}
How can we express the overage cost and and shortage costs in terms of the cost and price parameters introduced earlier?
\end{question}

\begin{solution}
In case of an overage; an item that is bought for $c$ was not sold and therefore salvaged at a price of $s$. The loss due to this item is therefore $h=c-s$. In case of a shortage; a unit of demand is not satisfied leading to a loss of $p$ from profit, but as no item was bought in the first place this loss is reduced by $c$. The net loss is thus $b=p-c$. Observe that both $h$ and $b$ are positive if $p>c>s$.
\end{solution}

\begin{question}
Once again, assume that the newsvendor buys $Q$ items and demand turns out to be $d$ items. Write an expression for the newsvendor's cost by using the overage and shortage costs.
\end{question}

\begin{solution}
\begin{align*}
h\max\{Q-d,0\} + b\max\{d-Q,0\}
\end{align*}
\end{solution}

The next exercise illustrates that the profit and cost formulations are equivalent, but they are not the same.

\begin{question}
Explain how the expressions you wrote for the profit (by using $p$, $c$, and $s$) and cost (by using $h$ and $b$) relate each other.
\end{question}

\begin{solution}
First, observe that (easy to verify; simply check two cases where $Q>d$ and $d>Q$)
\begin{align*}
\min\{Q,d\} = d-\max\{d-Q,0\} = Q-\max\{Q-d,0\}.
\end{align*}

If we make use of the above, then after some algebraic manipulation we obtain 
\begin{align*}
 -cQ + p\min\{Q,d\} + s(Q-\min\{Q,d\}) \\
% = -c(d-\max\{d-Q,0\}+\max\{Q-d,0\}) + p(d-\max\{d-Q,0\}) + s(Q-(Q-\max\{Q-d,0\})) \\
% = -cd+c\max\{d-Q,0\}-c\max\{Q-d,0\} + pd-p\max\{d-Q,0\} + s\max\{Q-d,0\} \\
% = (p-c)d + (c-p)\max\{d-Q,0\} + (s-c) \max\{Q-d,0\} \\
% = (p-c)d -\left((c-s) \max\{Q-d,0\} + (p-c)\max\{d-Q,0\}\right) \\
 = (p-c)d - \left(h\max\{Q-d,0\} + b\max\{d-Q,0\}\right).
\end{align*}

This tells us that the profit expression is equal to $(p-c)d$ minus the cost expression. Notice that, $(p-c)d$ stands for the maximum possible profit from sales (i.e. cannot sell more than the demand) and it is independent of the newsvendor's buying quantity. As such, the cost expression provides the loss from maximum possible profit.
\end{solution}

In the following, we want to use the notation $x^+$ and $x^-$ which works as follows: $x^+=\max\{x,0\}$ and $x^-=\max\{-x,0\}$. The next exercise illustrates a nice property of this notation.

\begin{question}
Show that $x=x^+-x^-$.
\end{question}

\begin{solution}
Let us consider two cases: $x\geq 0$ and $x<0$. In the first case, we have $x^+=\max\{x,0\}=x$ and $x^-=\max\{-x,0\}=0$. Thus $x=x-0$ holds. In the second case, we have $x^+=\max\{x,0\}=0$ and $x^-=\max\{-x,0\}=-x$. Thus $x=0-(-x)$ holds.
\end{solution}

\begin{question}
Write the cost expression again, but this time with the $x^+$ and $x^-$ notation.
\end{question}

\begin{solution}
\begin{align*}
h(Q-d)^+ + b(Q-d)^-
\end{align*}
\end{solution}

\begin{question}
Try  to rewrite the  expression for the profit in the form
\begin{equation}\label{eq:1}
P(Q) = (p -c)Q  + (s-p)(Q-d)^+.
\end{equation}
\end{question}

\begin{solution}
With the above  result that $\min\{Q,d\} = Q-\max\{Q-d,0\} = Q - (Q-d)^+$ we see that
\begin{equation*}
  \begin{split}
P(Q) 
&= (s-c)Q + (p-s) \min\{Q,d\}\\
&= (s-c)Q + (p-s) (Q- (Q-d)^+\\
&= (s-c)Q +  (p-s) Q - (p-s)(Q-d)^+\\
&= (s-c + p -s)Q  - (p-s)(Q-d)^+\\
&= (p -c)Q  - (p-s)(Q-d)^+ = (p -c)Q  + (s-p)(Q-d)^+.
  \end{split}
\end{equation*}
\end{solution}


\begin{question}\label{ex:nw_det}
Let $d=20$, $h=1$ and $b=10$; and plot the cost expression for different values of $Q$. 
\end{question}

\begin{solution}
See Figure~\ref{fig:cost_expression}.

\begin{figure}[htbp]
\centering
\begin{tikzpicture}
\scriptsize
\def\avg{20}
\def\h{1}
\def\p{10}
\begin{axis}[
	xlabel=$Q$,
	legend style={nodes=right},
	axis lines=middle,
	ymin=0]
\addplot [color=blue,thick,domain=0:2*\avg] {Ldeter(\x,\avg,\h,\p)};
\addlegendentry{$\h(Q - \avg)^+ + \p(Q - \avg)^-$}
\end{axis}
\end{tikzpicture}
\caption{Cost expression}
\label{fig:cost_expression}
\end{figure}
\end{solution}

\begin{question}
What is the newsvendor's best replenishment quantity in the previous exercise? What would be her cost if she replenishes this amount? How would her cost be effected if she were buy one unit less or one unit more than this quantity?
\end{question}

\begin{solution}
The optimal quantity is obviously 20, a we know that the demand is 20 units anyway. The newsvendor's cost will be zero if she replenishes this quantity. Having a unit more will exactly increase the cost by \$1 and a unit less by \$10.
\end{solution}

\begin{question}
Let newsvendor's selling price be \$15 in the previous exercise. Write down the profit and cost expressions for replenishment quantities $\{7,26\}$ and compare the results.
\end{question}

\begin{solution}
Because $h=c-s=1$ and $b=p-c=10$, and as $p=15$; we have that $c=5$ and $s=4$. Thus, we obtain the profit expression
\begin{align*}
-cQ + p\min\{Q,d\} + s(Q-\min\{Q,d\}) = -5\cdot Q + 15\cdot\min\{Q,20\} + 4\cdot(Q-\min\{Q,20\})
\end{align*}
and the cost expression
\begin{align*}
h(Q-d)^+ + b(Q-d)^- = 1\cdot (Q-20)^+ + 10\cdot(Q-20)^-.
\end{align*}

Then; for $Q=7$ we obtain profit \$70 and cost \$130, and for $Q=26$ we have \$194 for profit and \$6 for cost. Indeed, in botch cases, profit and cost expressions sum up to the maximum possible profit $(p-c)d=(15-5)\cdot 20=200$.
\end{solution}

We now begin investigating the newsvendor problem with stochastic demand. To that end, we replace the deterministic demand $d$ with the random demand $D$. 

\begin{question}
Let $L(Q)$ be the expected total cost of the newsvendor given that she buys $Q$ items. Write an expression for this function.
\end{question}

\begin{solution}
\begin{align*}
L(Q) = h\ex (Q-D)^+ + b\ex (Q-D)^-
\end{align*}
Here, we simply used an expectation operator. How we approach this operator depends on the demand distribution. 
\end{solution}

\begin{question}
How $L(Q)$ can be computed?
\end{question}

\begin{solution}
It depends on the demand distribution. 

If demand distribution is continuous than we have 
\begin{align*}
L(Q) & = h \ex (Q-D)^+ + b \ex (Q-D)^- \\
	 & = \int_x \left(h (Q-x)^+ + b (Q-x)^-\right) f(x) \dif x
\end{align*}
where $f(x)$ is the probability density function of $D$. This can be computed easily (in closed form) for some distributions (e.g. normal and lognormal distributions), whereas for some others numerical integration could be quite involved computatinally.

If demand distribution is discrete then we have 
\begin{align*}
L(Q) & = h \ex (Q-D)^+ + b \ex (Q-D)^- \\
	 & = \sum_x \left(h (Q-x)^+ + b (Q-x)^-\right) f(x)
\end{align*}
where $f(x)$ is the probability mass function of $D$. This cannot be computed in closed form in general. But it's numerical computation is trivial and it can easily be done in spreadsheet applications.
\end{solution}


We now concentrate on the computational procedure for normally distributed demand. 

\begin{question}
How $L(Q)$ can be computed if $D$ is normally distributed?
\end{question}

\begin{solution}
Let $\mu$ denote the mean and $\sigma$ the standard deviation of $D$. Then (omitting the math) we have
\begin{align*}
L(Q) 
& = h \ex (Q-D)^+ + b \ex (Q-D)^- \\
& = h(Q-\mu) + (h+b) \sigma F_{\text{loss}}(z).
\end{align*}
where $z=\frac{Q-\mu}{\sigma}$ and $F_{\text{loss}}$ is the 'standard normal loss function'. 

The standard normal loss function is defined as 
\begin{align*}
F_{\text{loss}}(z) = \phi(z)-z(1-\Phi(z))
\end{align*}
where $\phi(z)$ and $\Phi(z)$ are 'standard normal density function' and 'standard normal distribution function' respectively. 

These functions are accessible in spreadsheet applications. For instance, in Excel we have 
\begin{align*}
\phi(z) \text{=NORMDIST($z$,0,1,FALSE)}
\end{align*}
and
\begin{align*}
\Phi(z) \text{=NORMSDIST($z$)}.
\end{align*}

Therefore, we can compute the 'standard normal loss function as
\begin{align*}
F_{\text{loss}}(z) \text{=NORMDIST($z$,0,1,FALSE)-$z$*(1-NORMSDIST($z$))}.
\end{align*}
\end{solution}

\begin{question}\label{ex:nw_stoc}
Let $D$ be normally distributed with mean 20 and standard deviation 5, $h=1$ and $b=10$; and plot $L(Q)$ for different values of $Q$. 
\end{question}

\begin{solution}
See Figure~\ref{fig:LQ_normal}.

\begin{figure}[htbp]
\centering
\begin{tikzpicture}
\scriptsize
\def\avg{20}
\def\std{5}
\def\h{1}
\def\p{10}
\begin{axis}[
	xlabel=$Q$,
	legend style={nodes=right},
	axis lines=middle,
	ymin=0]
\addplot [color=red,thick,domain=0:2*\avg] {Lstoch(\x,\avg,\std,\h,\p)};
\addlegendentry{$\h\ex(Q - D)^+ + \p\ex(Q - D)^-$}
\end{axis}
\end{tikzpicture}
\caption{$L(Q)$ for normally distributed demand}
\label{fig:LQ_normal}
\end{figure}
\end{solution}


\begin{question}
Have a look at the graphs you obtained in Exercise~\ref{ex:nw_det} and Exercise~\ref{ex:nw_stoc}. What similarities and differences you see?
\end{question}

\begin{solution}
See Figure~\ref{fig:LQ_comparison}. 

\begin{enumerate}
\item The stochastic cost is always larger than the deterministic one. this is due to the fact that there is always a non-zero probability of overage and shortage in the stochastic case. 
\item For very small and very large values of order quantity cost figures coincide. This can be explained as follows. On the one hand, if order quantity is very large then no demand will be lost. Therefore, the expected total cost will be overage cost times the order quantity minus the expected demand, i.e. $h(Q-\ex D)$. That's why the we have $L(40)\approx 1\cdot(40-20)=20$. This is the very same cost that would realize if demand was exactly 20 units and the order quantity was 40. On the other hand, if the order quantity is very small then all demand will be lost. Therefore, the expected total cost will be shortage cost times the expected demand, i.e. $b \ex D$. That's why the we have $L(0)\approx 10\cdot20=200$. This is the very same cost that would realize if demand was exactly 20 units and the order quantity was 0.
\item It is optimal to order 20 units in the determinstic case , whereas it is a bit higher in the stochastic case. This can be attributed to the demand uncertainty and shortages being more expensive as compared to overages. 
\end{enumerate}


\begin{figure}[htbp]
\centering
\begin{tikzpicture}
\scriptsize
\def\avg{20}
\def\std{5}
\def\h{1}
\def\p{10}
\begin{axis}[
	xlabel=$Q$,
	legend style={nodes=right},
	axis lines=middle,
	ymin=0]
\addplot [color=blue,thick,domain=0:2*\avg] {Ldeter(\x,\avg,\h,\p)};
\addplot [color=red,thick,domain=0:2*\avg] {Lstoch(\x,\avg,\std,\h,\p)};
\addlegendentry{$\h(Q - \avg)^+ + \p(Q - \avg)^-$}
\addlegendentry{$\h\ex(Q - D)^+ + \p\ex(Q - D)^-$}
\end{axis}
\end{tikzpicture}
\caption{A comparison of costs for deterministic and stochastic demands}
\label{fig:LQ_comparison}
\end{figure}
\end{solution}


\begin{question}
Let us consider cases where demand is always normally distributed with mean 20 but with different standard deviation levels $\{1,5,10\}$. Plot $L(Q)$ for these cases assuming $h=1$ and $b=10$. What do you observe?
\end{question}

\begin{solution}
See Figure~\ref{fig:LQ_std}. We observe that increasing standard deviation leads to further deviations from the deterministic case and larger optimal order quantities. 

\begin{figure}[htbp]
\centering
\begin{tikzpicture}
\scriptsize
\def\avg{20}
\def\h{1}
\def\p{10}
\begin{axis}[
	xlabel=$Q$,
	legend style={nodes=right},
	axis lines=middle,
	ymin=0]

\addplot [color=red,thick,domain=0:2*\avg] {Lstoch(\x,\avg,1,\h,\p)};
\addlegendentry{$\sigma=1$}

\addplot [color=blue,thick,domain=0:2*\avg] {Lstoch(\x,\avg,5,\h,\p)};
\addlegendentry{$\sigma=5$}

\addplot [color=green,thick,domain=0:2*\avg] {Lstoch(\x,\avg,10,\h,\p)};
\addlegendentry{$\sigma=10$}

\end{axis}
\end{tikzpicture}
\caption{$L(Q)$ with different levels of standard deviation}
\label{fig:LQ_std}
\end{figure}
\end{solution}

\begin{question}
How can we compute the optimal order quantity if demand is normally distributed?
\end{question}

\begin{solution}
Because we know how to compute the cost function for different values of $Q$, we can try many reasonable values and find the best. 

Nevertheless, it is also possible to compute it in closed form. The optimal order quantity (omitting the math) can be computed as 
\begin{align*}
Q^* 
& = F^{-1}\left(\frac{b}{h+b}\right) \\
& = \mu + \sigma \Phi^{-1}(\beta) 
\end{align*}
where $\Phi^{-1}(\beta)$ is the 'inverse standard normal distribution' evaluated at $\beta=\frac{b}{h+b}$. 

The inverse standard normal distribution can easily be computed in spreadsheet applications. For instance, in Excel we have 
\begin{align*}
\Phi^{-1}(\beta)\text{=NORMSINV($\beta$)}.
\end{align*}
\end{solution}

\begin{question}
Let $D$ be normally distributed with mean 20 and standard deviation 5, $h=1$ and $b=10$. What is the optimal order quantity? Verify your result by comparing it against the graph you obtained in Exercise~\ref{ex:nw_stoc}. 
\end{question}

\begin{solution}
\begin{align*}
Q^* 
& = \mu + \sigma \Phi^{-1}(\beta) \\
& = 20 + 5\cdot \phi^{-1}(10/11) = 20 + 5\cdot \Phi^{-1}(1.3352) = 26.676
\end{align*}
\end{solution}

\begin{question}
We call 'safety stock' 

Have a closer look at the expression of the optimal order quantity. How does it affected by the mean
\end{question}

\begin{solution}
•
\end{solution}

%Next, we consider arbitrary discrete demand distributions. 
%
%\begin{question}
%How $L(Q)$ can be computed if $D$ has a discrete distribution function $F(x)$?
%\end{question}
%
%\begin{solution}
%Let $\mu$ denote the mean of $D$. Then (omitting the math) we have
%\begin{align*}
%L(Q) 
%& = h\ex (Q-D)^+ + b\ex (Q-D)^- \\
%%& = h\sum_{x=0}^Q (Q-x) + b \sum_{x=Q+1}^{\infty} (x-Q) \\  
%& = h(Q-\ex D) + (h+b)\ex (Q-D)^- \\
%\end{align*}
%
%\end{solution}



%\begin{align*}
%L(Q) = \sum_x h\left((Q-x)^+ + b\ex (Q-x)^-\right) f(x)\dif x.
%\end{align*}
%
%
%whether demand has a continuous distribution (e.g. normal distribution) or discrete distribution (e.g. Poisson distribution). 
%
%If demand has a continuous distribution we have
%\begin{align*}
%L(Q) = \int_x h\left((Q-x)^+ + b\ex (Q-x)^-\right) f(x)\dif x.
%\end{align*}
%where $f(x)$ is the density function of $D$. 
%
%If demand has a discrete distribution we have
%\begin{align*}
%L(Q) = \sum_{x=0}^{\infty} h\left((Q-x)^+ + b\ex (Q-x)^-\right) f(x)\dif x.
%\end{align*}
%where $f(x)$ is the density function of $D$. 



